% Classe modercv
% Article complet : http://blog.madrzejewski.com/creer-cv-elegant-latex-moderncv/

\documentclass[11pt,a4paper]{moderncv}
\usepackage[utf8]{inputenc}
\usepackage[frenchb]{babel}
\usepackage[official]{eurosym}
\usepackage[top=1.5cm, bottom=1.5cm, left=1.5cm, right=1.5cm]{geometry}

\usepackage{xcolor}
\definecolor{airforceblue}{rgb}{0.19, 0.55, 0.91}

%-------------------------- Options package "moderncv"  ------------------------------%
\moderncvtheme[blue]{classic} % [blue,orange,green,red,purple,grey,black] {classic,casual,oldstyle,banking} 
% Largeur de la colonne pour les dates
\setlength{\hintscolumnwidth}{2.5cm}
\firstname{Mateusz}
\familyname{Bogdan}
\title{Normalien, Docteur -- Génie Civil}              
\address{47, rue de la Beaune}{93100 Montreuil}    
\email{mat.nadgobz@gmail.com}
\homepage{matbog.github.io}
\mobile{+33 6 47 50 43 55} 
\extrainfo{Nationalité française\\ 36 ans -- Célibataire}
\photo[80pt][1pt]{cv_pictureBW.eps} % [hauteur][epaisseur cadre]

%-------------------------------------------------------------------------------------%
\begin{document}
\maketitle


%-------------------------------------------------------------------------------------%
\section{Expériences professionnelles}

\cventry{2009 - 2014}{Enseignant}{ENS Cachan, ESTP, CNAM, Vietnam}{}{}{
Travaux dirigés, travaux pratiques et cours magistraux. 
\begin{itemize}
\item Thermique, Thermique du bâtiment, Acoustique, Résistance des matériaux, Construction en béton armé, Matériaux Cimentaires.
\item Supervision de stage Licence 3 (ENS Cachan) \og \'Etude de la diffusion accélérée des chlorures  par champs électrique  dans les matériaux cimentaires \fg{}.
\item Jury de soutenance de stage Master 2 Génie Civil (ENS Cachan).
\end{itemize}}

\cventry{mars - juin 2011}{Stage de Master 2}{LMT Cachan}{Cachan}{}{Modélisation morphologique de matériaux cimentaires et de leur hydratation}{} %\newline{}}

\cventry{été 2010}{Interprète}{Safran - Hispano Suiza}{Colombe}{}{Franco-polonais, Interprète - Traducteur }{} %\newline{}}

\cventry{2009 -- 2010}{Directeur Scientifique}{Vietnam Green Building Council}{Hanoï}{Vietnam}{
\begin{itemize}
	\item Supervision d'une équipe de trois personnes.
	\item \'Evaluation et reprise du système de certification \textsc{Lotus} 
	\begin{itemize}
		\item Efficacité des systèmes énergétiques, de la gestion de l'eau et des ressources,
		\item Protection de la santé des occupants,
		\item Minimisation des déchets, de la pollution et des dégradations environnementales.
	\end{itemize}
	\item Publication de la version pilote du système de certification pour les bâtiments non résidentiels : LOTUS NR.
\end{itemize}}{} %\newline{}}

\cventry{avril - juin 2008}{Stage de Master 1}{University of California}{Berkeley}{\'Etats Unis}{
Determination numérique et expérimentale des fréquences propres d'une dalle d'essai sismique}{}

\cventry{été 2007}{Stage en entreprise}{Eiffage - BG 21 Ingénieurs Conseils }{Paris}{}{Réalisation d'études de sécurités vis-à-vis des risques d'explosions de gaz sur une station d'épuration} %\newline{}}

%---------------------------------------------------------------------%
\section{Formation}

\cventry{2012 -- 2015}{Doctorat}{LMT Cachan -- ENS Cachan}{}{}{Modélisation morphologique multi-échelles de matériaux cimentaires -- Prédiction de propriétés de diffusion effectives 
\begin{itemize}
	\item Développement d'un modèle d'hydratation des pâtes de ciment.
	\item Adaptation d'un modèle morphologique aux différentes échelles rencontrées dans les matériaux cimentaires.
	\item Campagnes de simulations numériques des transferts de masse dans les matériaux poreux, homogénéisation multi-échelles de propriétés effectives.
\end{itemize}
}

\cventry{2010 -- 2011}{Master 2 Recherche}{ENS Cachan}{}{}{ \og Génie Civil et Environnement -- Sciences de l'ingénieur \fg -- \textit{mention Bien}}

\cventry{2008 -- 2009}{Agrégation de Génie Civil}{Admissible}{}{}{option \og Matériaux, Ouvrages et Aménagement \fg}

\cventry{2006 -- 2008}{Licence 3 - Master 1}{ENS Cachan}{}{}{Spécialité Génie Civil}



\section{Compétences}

\cvlanguage{Anglais}{Professionnel -- Toeic 850 (2007)}{Expérience professionnelle internationale d'un an (2010) et environnement international durant les trois ans de thèse}

\cvlanguage{Polonais}{Langue maternelle}{}

%\cvlanguage{Allemand}{Débutant}{}

\cvitem{Langages}{ C/C++, Pearl, Bash, Fortran, HTML-CSS/PHP}

\cvitem{Logiciels}{Cat3m, Feap, Catia, Matlab/Scilab, Paraview, Microsoft Office, Open Office, Suite Adobe, ...}

\cvitem{Secourisme}{Formation Premiers Secours (2005)}

%-------------------------------------------------------------------------------------%
\section{Communications scientifiques}

%--------------------------------------%
\subsection{Articles}

\cventry{2015}{Defect and Diffusion Forum}{Defect and Diffusion Forum}{Vol. 362}{Effective Permeability and Transfer Properties in Fractured Porous Media}{G. Rastiello, R. Bennacer, G. Nahas, \textsl{M. Bogdan}}

\cventry{}{European Journal of Environmental and Civil Engineering}{Vol. 19}{}{Cement Paste Morphologies and effective diffusivity Using the Lattice Boltzmann Method}{E. Walther, \textsl{M. Bogdan}, C. Desa, R. Bennacer}

\cventry{}{Water Ressources Research}{Special issue: Applications of percolation theory to porous media}{\textbf{soumis}}{Continuous percolation on finite size domains: Use of the Excursion set theory}{E. Roubin, \textsl{M. Bogdan}, L. Stefan, J.-B. Colliat, F. Benboudjema}
%--------------------------------------%
\subsection{Conférences}

\cventry {2011}{COMPLAS XI}{Barcelone}{Espagne}{Multi-scale failure for heterogeneous materials: link with morphological modeling}{E. Roubin, \textsl{M. Bogdan}, J.-B. Colliat}

\cventry{2012}{Transfert 2012}{Lille}{}{Modélisation des transferts de masse dans les matériaux à matrice cimentaire à l’aide d’un modèle morphologique}{\textsl{M. Bogdan}, E. Roubin, J.-B. Colliat, F. Benboudjema, L. Stefan}

\cventry{}{Microdurability 12}{Amsterdam}{Pays Bas}{Morphological modelling of cement based materials and hydration process}{\textsl{M. Bogdan}, E. Roubin, J.-B. Colliat, F. Benboudjema, L. Stefan}

\cventry{2013}{AUGC'13}{ENS Cachan}{}{Modélisation des transferts diffusifs dans les matériaux cimentaires}{\textsl{M. Bogdan}, J.-B. Colliat, F. Benboudjema, L. Stefan}

\cventry{2014}{AUGC'14}{Polytech Orléans}{}{Modèle d’hydratation \& Méthodes Level-Set}{\textsl{M. Bogdan}, J.-B. Colliat, F. Benboudjema, L. Stefan}

\cventry{}{AUGC'14}{Polytech Orléans}{}{Morphologies et homogénéisation par la Lattice Boltzmann
Method}{E. Walther,\textsl{M. Bogdan}, R. Bennacer, C. Desa}

\cventry{}{Euro-C}{St. Anton am Arlberg}{Autriche}{Microscopic model for concrete diffusivity prediction}{\textsl{M. Bogdan}, J.-B. Colliat, F. Benboudjema, L. Stefan}

%-----------------------------------------------------------%
\section{Activités associatives}

\cventry{2013}{AUGC'13}{ENS Cachan}{}{}{
\begin{itemize}
\item Création du site web (\textsc{hmtl-css}),
\item Membre de l'équipe organisatrice des rencontres de l'Association Universitaire de Génie Civil.
\end{itemize}}

\cventry{2011 -- 2012}{Président de l'association \og Le Parasol\fg{}}{LMT Cachan}{}{}{En charge de l'approvisionnement de la cafétéria (budget annuel de 15k\euro)}

\cventry{2006 -- 2007}{Membre du bureau du BDE}{ENS Cachan}{Responsable \og K-fet\fg}{}{Co-gestionnaire général de la maison des étudiants (budget annuel estimé à 250k\euro)}

\cventry{... -- 2002}{Scoutisme}{Lisses - Courcouronne, Sceaux}{}{}{Du plus jeune âge jusqu'à l'encadrement.}


%-------------------------------------------------------------------------------------%
\section{Centres d'intérêt}
\cvitem{Architecture}{Intérêt particulier pour l'architecture moderne et contemporaine.}
\cvitem{Théâtre}{Classiques de la littérature européenne, jeunes auteurs internationaux, suivi de metteurs en scène reconnus ou prometteurs}
\cvitem{Sport}{Hockey sur glace, escalade, golf, cyclisme}

\end{document}








%% plein de choses différentes pour présenter vos compétences
%\cvitemwithcomment{Compétence}{Niveau}{Commentaire}
%\cvdoubleitem{Categorie 1}{XXX, YYY, ZZZ}{Categorie 2}{XXX, YYY, ZZZ}
%\cvitem{Quelque chose}{Description}
%% une liste
%\cvlistitem{Item 1}
%\cvlistitem{Item 2}
%\cvlistitem{Item 3}
%% une double liste
%\cvlistdoubleitem{Item 1}{Item 3}
%\cvlistdoubleitem{Item 2}{Item 4}